\documentclass[11pt]{jarticle} 

% 絶体に必要なおまじない
% %文字以降はコメントとして無視される
% 11pt は標準の文字の大きさの指定
% 10pt とすれば小さく、12pt とすれば大きくなる

\usepackage[dvipdfmx]{graphicx} % 画像を使うためのパッケージ
\usepackage{amsmath, amsfonts} % 数式を使うためのパッケージ
\usepackage{bm} % 太字の数式を使う
\usepackage{fancyhdr}

\setlength{\oddsidemargin}{-10mm} % 紙の左のマージン
\setlength{\evensidemargin}{-10mm} % 紙の右のマージン
\setlength{\textwidth}{180mm} % テキストの幅 を指定

\setlength{\topmargin}{-15mm} % 紙の上のマージン
\setlength{\textheight}{255mm} % テキストの高さ

\lhead{東京科学大学 中間報告書} %ヘッダ左
\chead{\empty} %ヘッダ中央
\rhead{2025年8月1日提出} %ヘッダ右.コンパイルした日付を表示
\lfoot{\empty} %フッタ左
\cfoot{\thepage} %フッタ中央.ページ番号を表示
\rfoot{\empty} %フッタ右

\pagestyle{fancy}

% ここから本文
\begin{document}
\title{段階的制御器合成におけるminimiseを用いた\\計算空間削減に関する研究}
\author{山口 友輝 (22B30862) \;\; 指導教員:鄭 顕志}
\date{\empty} % 日付は自動的に今日の日付が入る
\maketitle % 上で指定した title と authorを出力
\thispagestyle{fancy}

\section{概要} 
本研究では,離散制御器合成において避けがたい問題である状態空間の爆発を抑制する手法として,段階的制御器合成とminimiseの併用に着目する.段階的制御器合成は,複数の要求を段階的に環境に適用することで,状態空間を局所的に抑えながら合成を行う手法である.本研究では,その各段階の末尾でminimiseを適用することで,状態空間を小さく保ちつつ合成できるかを検証する.

\section{研究背景}
離散事象システム$^{[1]}$は,アクションの発火により状態が離散的に遷移するシステムであり,環境モデル$^{[1]}$や監視モデル$^{[1]}$といったラベル付き状態遷移システム(LTS)として形式的に表現される.ここで,環境モデルはシステムの動作環境をLTSで表現し,監視モデルはシステムが満たす安全性をLTSで表現したものである.このようなシステムに対して,環境モデルと監視モデルを入力として,これらを自動で合成し.制御器を出力する離散制御器合成$^{[1]}$という技術がある.制御器とは監視モデルの安全性を保証しながら,環境モデルの動作を制御するようにLTSで表現されたものである.\\
\indent
離散制御器合成は,システムの設計や自動化において有効な手段であり,システムが予測不可能な環境にあるときに自身で対処するするシステムである自己適応システムの実現する手法であり,近年では,Model Transition System Analyzer(MTSA)といった形式手法ツールを用いた合成実験も進んでおり,実用的な応用が可能となりつつある.

\section{既存研究における課題}
離散制御器合成では,入力となる環境モデルや要求モデルが複数ある場合,それらを同時に合成することによって状態空間が指数的に増加する問題がある.この状態爆発は,制御器の合成が計算上困難になる主な要因となっている.\\
\indent
例えば,飛行機の離着陸制御や部屋への出入り管理のようなモデルでは,構成要素の数が増えるごとに状態数が爆発的に増加し,ツールによる合成処理が不可能になる場合がある.

\section{提案手法}
本研究では,状態爆発の抑制を目的として,段階的制御器合成における各段階の末尾でminimiseを適用する手法を提案する.\\
\indent
段階的制御器合成$^{[2]}$とは,複数の監視モデルを一度に環境モデルと合成するのではなく,各監視モデルごとに注目して安全性を保証しながら部分的に合成し,これによって合成された部分制御器を以降入力として扱い新たな部分制御器を合成するもので,これをすべての監視モデルが環境モデルに反映されるまで行う.合成対象の状態数を局所的に抑える手法である.この手法により,制御器を合成する過程で構築される違反状態を取り除いていない全体モデルの状態数を少なくすることができ,状態空間を小さくすることができる.\\
\indent
一方,minimiseは,生成されたLTSに対して同値な状態や遷移をまとめることによって状態数を削減する操作であり,合成後の状態空間を縮小する効果がある.\\
\indent
本研究の提案手法では,段階的制御器合成の各ステップで部分的に合成された部分制御器に対してminimiseを適用し,次の要求の合成時にminimiseによって状態が少なくなった部分制御器を使用する.これにより,最終的な制御器全体の状態数や遷移数削減することを目指す.\\



% \begin{figure}[htbp]
%   \begin{center}
%   \includegraphics[width=15cm]{SDCS_minimise.png}
%   \caption{段階的制御器合成におけるminimise操作の適用例}
%   \label{fig:1}
%   \end{center}
% \end{figure}


\section{研究計画}
8月 段階的制御器合成とminimiseを併用する実装,実験\\
\indent
9月 国内学会論文投稿\\
\indent
11月 国内学会発表\\

\appendix % Appendix はこのようにして始まる
\begin{thebibliography}{99} % 文献リスト一覧はこんな風に始める

\bibitem{magee} 
J.~Magee and J.~Kramer, 
{\it Concurrency: State Models and Java Programs}, 2006.

\bibitem{yamauchi} 
山内拓人, 鄭顕志, 
「段階的な部分合成による離散制御器合成の分析空間削減」, 
電子情報通信学会論文誌 D, Vol.~J106-D, No.~4, pp.~218--230, 2023.

\end{thebibliography} % 文献リストの最後に必要

\end{document} % 最後の締めくくり。絶体必要